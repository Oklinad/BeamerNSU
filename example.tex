\documentclass[8pt, hyperref={pdftex,unicode}, green]{beamer}
\usetheme{NSU}


\usepackage{helvet}
\usepackage{tikz}


\usepackage{colortbl}
\newcommand{\cc}{\bf\cellcolor{green}}

% --------------------------------------------------- %
%                       Информация   	              %
% --------------------------------------------------- %

\institute[НГУ]{
    Федеральное государственное автономное образовательное учреждение высшего образования\\ 
    «Новосибирский национальный исследовательский государственный университет»
}
\title[Восстановление латинских квадратов]{
    Анализ и восстановление комбинаторных объектов с~недостающими элементами на примере латинских~квадратов
}
\author[Данилко В. Р.]{Данилко Виталий Романович}

\date{7 июля 2020 г.}

\newcommand{\LSfive}[1]{
    \begin{array}{|>{\hspace{-2pt}}c<{\hspace{-2pt}}|>{\hspace{-2pt}}c<{\hspace{-2pt}}|>{\hspace{-2pt}}c<{\hspace{-2pt}}|>{\hspace{-2pt}}c<{\hspace{-2pt}}|>{\hspace{-2pt}}c<{\hspace{-2pt}}|}\hline
        #1
    \end{array}
}

\graphicspath{{figuras}}


% --------------------------------------------------- %
%                  Титульник + Содержание             %
% --------------------------------------------------- %

\begin{document}

{
\logo{}
\begin{frame}
    \maketitle
    \begin{minipage}{\linewidth}
      \hspace{0.25\linewidth}
      \begin{minipage}{0.55\linewidth}\small
        \textbf{Научные руководители:}\\
        Кочетов Юрий Андреевич, д.\,ф-м.\,н.\\
        Горкунов Евгений Владимирович, к.\,ф-м.\,н.\vspace{1em}\\
        \textbf{Студент:}\\
        Данилко Виталий Романович\\ 
        НГУ ФИТ, группа 18227
      \end{minipage}
    \end{minipage}
\end{frame}
}

% \begin{frame}{Содержание}
%   \tableofcontents
% \end{frame}

% --------------------------------------------------- %
%                      Презентация                    %
% --------------------------------------------------- %

\section{Актуальность исследования}
\begin{frame}{Актуальность исследования}

    \begin{block}{Связанные проблемы}
        \begin{itemize}
            \item Восстановление квадратов
            \item Пересечение кодов
            \item Расстояние между функциями
        \end{itemize}
    \end{block}
    
\end{frame}

\section{Задачи}
\begin{frame}{Задачи}

    \begin{block}{Задачи}
        \begin{itemize}
            \item Изучение и анализ свойств системы попарно ортогональных латинских квадратов
            \item Разработка алгоритма восстановления системы ортогональных латинских квадратов
            \item Обоснование разработанного алгоритма
            \item Увеличение производительности алгоритма
            \item Анализ разработанного алгоритма
        \end{itemize}
    \end{block}

\end{frame}

\section{Определения}
\subsection{Латинский квадрат}
\begin{frame}{Определения}

    \begin{block}{Латинский квадрат}
        {\it Латинский квадрат порядка $n$} — это таблица $L = (l_{ij})$ размеров $n\times n$, заполненная $n$ элементами множества $M$ таким образом, что в каждой строке и в каждом столбце таблицы каждый элемент из $M$ встречается в точности один раз.
    \end{block}
    
    $$
    \LSfive{0 & 1 & 2 & 3 & 4 \\ \hline
            4 & 0 & 1 & 2 & 3 \\ \hline
            3 & 4 & 0 & 1 & 2 \\ \hline
            2 & 3 & 4 & 0 & 1 \\ \hline
            1 & 2 & 3 & 4 & 0 \\ \hline}
    $$
\end{frame}

\subsection{Ортогональность}
\begin{frame}{Определения}

    \begin{block}{Ортогональность}
        Два латинских квадрата $L = (l_{ij})$ и $K = (k_{ij})$ порядка $n$ называются {\it ортогональными}, если все $n^2$ упорядоченных пар $(l_{ij}, k_{ij})$ различны.
    \end{block}
    
    $$
    \LSfive{
        0 & 1 & 2 \\ \hline
        1 & 2 & 0 \\ \hline
        2 & 0 & 1 \\ \hline
    }
    \hspace{2em}
    \LSfive{
        0 & 1 & 2 \\ \hline
        2 & 0 & 1 \\ \hline
        1 & 2 & 0 \\ \hline
    }
    \hspace{2em}
    \LSfive{
        (0, 0) & (1, 1) & (2, 2) \\ \hline
        (1, 2) & (2, 0) & (0, 1) \\ \hline
        (2, 1) & (0, 2) & (1, 0) \\ \hline
    }
    $$
    
    \begin{block}{}
        Полная система содержит $n-1$ попарно ортогональных латинских квадратов порядка $n$.
    \end{block}
    

\end{frame}


\section{Гипотеза и теорема}
\begin{frame}{Гипотеза и теорема}

    \begin{block}{Гипотеза (С. В. Августинович, ИМ СО РАН)}
        Полная система попарно ортогональных латинских квадратов порядка~$n$ при наличии не более $2n-1$ неизвестных элементов в каждом из квадратов однозначно восстанавливается.
    \end{block}
    
    \begin{block}{Теорема}
        Полная система попарно ортогональных латинских квадратов порядка $n \in \{3,4,5,7,8\}$ при наличии не более $2n-1$ неизвестных элементов в каждом из квадратов однозначно восстанавливается. 
    \end{block}
  
\end{frame} 


\section{Система Боуза}
\begin{frame}{Система Боуза}
    
    \begin{block}{}
    \begin{gather*}
        f(x,y)= \alpha x + y, \quad \alpha \in F_p^*\\
        \begin{array}{|>{\hspace{-2pt}}c<{\hspace{-2pt}}|>{\hspace{-2pt}}c<{\hspace{-2pt}}|>{\hspace{-2pt}}c<{\hspace{-2pt}}|>{\hspace{-2pt}}c<{\hspace{-2pt}}|>{\hspace{-2pt}}c<{\hspace{-2pt}}|} 
            \multicolumn{5}{c}{\alpha = 1} \\ \hline
            0 & 1 & 2 & 3 & 4 \\ \hline
            4 & 0 & 1 & 2 & 3 \\ \hline
            3 & 4 & 0 & 1 & 2 \\ \hline
            2 & 3 & 4 & 0 & 1 \\ \hline
            1 & 2 & 3 & 4 & 0 \\ \hline
        \end{array}\hspace{1em}
        \begin{array}{|>{\hspace{-2pt}}c<{\hspace{-2pt}}|>{\hspace{-2pt}}c<{\hspace{-2pt}}|>{\hspace{-2pt}}c<{\hspace{-2pt}}|>{\hspace{-2pt}}c<{\hspace{-2pt}}|>{\hspace{-2pt}}c<{\hspace{-2pt}}|} 
            \multicolumn{5}{c}{\alpha = 2} \\ \hline
            0 & 1 & 2 & 3 & 4 \\ \hline
            3 & 4 & 0 & 1 & 2 \\ \hline
            1 & 2 & 3 & 4 & 0 \\ \hline
            4 & 0 & 1 & 2 & 3 \\ \hline
            2 & 3 & 4 & 0 & 1 \\ \hline
        \end{array}\hspace{1em}
        \begin{array}{|>{\hspace{-2pt}}c<{\hspace{-2pt}}|>{\hspace{-2pt}}c<{\hspace{-2pt}}|>{\hspace{-2pt}}c<{\hspace{-2pt}}|>{\hspace{-2pt}}c<{\hspace{-2pt}}|>{\hspace{-2pt}}c<{\hspace{-2pt}}|} 
            \multicolumn{5}{c}{\alpha = 3} \\ \hline
            0 & 1 & 2 & 3 & 4 \\ \hline
            2 & 3 & 4 & 0 & 1 \\ \hline
            4 & 0 & 1 & 2 & 3 \\ \hline
            1 & 2 & 3 & 4 & 0 \\ \hline
            3 & 4 & 0 & 1 & 2 \\ \hline
        \end{array}\hspace{1em}
        \begin{array}{|>{\hspace{-2pt}}c<{\hspace{-2pt}}|>{\hspace{-2pt}}c<{\hspace{-2pt}}|>{\hspace{-2pt}}c<{\hspace{-2pt}}|>{\hspace{-2pt}}c<{\hspace{-2pt}}|>{\hspace{-2pt}}c<{\hspace{-2pt}}|} 
            \multicolumn{5}{c}{\alpha = 4} \\ \hline
            0 & 1 & 2 & 3 & 4 \\ \hline
            1 & 2 & 3 & 4 & 0 \\ \hline
            2 & 3 & 4 & 0 & 1 \\ \hline
            3 & 4 & 0 & 1 & 2 \\ \hline
            4 & 0 & 1 & 2 & 3 \\ \hline
        \end{array}
    \end{gather*}
    \end{block}
  
\end{frame}


\section{Методы восстановления}
\subsection{Метод строк и столбцов}
\begin{frame}{Метод строк и столбцов}

    \begin{block}{}
        В каждом латинском квадрате системы ищем строку и столбец, содержащий только один неизвестный элемент. 
        Повторяем, пока любая строка и любой столбец не будут содержать либо 0, либо не менее 2 неизвестных элементов.
    \end{block}

    \begin{gather*}
        \LSfive{
             &  &  & 3 & 4 \\ \hline
             &  &  & 4 & 0 \\ \hline
            2 &  & 4 & \cc & \cc \\ \hline
            3 & 4 & 0 & 1 & 2 \\ \hline
            4 & 0 & 1 & 2 & 3 \\ \hline
        }
        \Rightarrow
        \LSfive {
             &  &  & 3 & 4 \\ \hline
             &  &  & 4 & 0 \\ \hline
            2 & \cc & 4 & 0 & 1 \\ \hline
            3 & 4 & 0 & 1 & 2 \\ \hline
            4 & 0 & 1 & 2 & 3 \\ \hline
        }
        \Rightarrow
        \LSfive{
             &  &  & 3 & 4 \\ \hline
             &  &  & 4 & 0 \\ \hline
            2 & 3 & 4 & 0 & 1 \\ \hline
            3 & 4 & 0 & 1 & 2 \\ \hline
            4 & 0 & 1 & 2 & 3 \\ \hline
        }
    \end{gather*}
\end{frame}
  

\subsection{Диагональный метод}
\begin{frame}{Диагональный метод}

    \begin{block}{}
        Ищем в каждом латинском квадрате системы диагонали размера $n - 1$.
        В других квадратах системы это же множество клеток образует трансверсаль, которую мы тоже можем однозначно восстановить, заполнив неизвестную клетку элементом, отсутствующим в рассматриваемой трансверсали.
    \end{block}

    \begin{gather*}
        \LSfive{
            0 & 1 &  &  &  \\ \hline
            1 & \cc 2 &  &  &  \\ \hline
            \cc 2 & 3 &  &  &  \\ \hline
            3 & 4 & 0 & 1 & \cc 2 \\ \hline
            4 & 0 & 1 & \cc 2 & 3 \\ \hline
        }
        \hspace{1.5em}
        \LSfive{
            0 & 1 &  &  &  \\ \hline
            3 & \cc 4 &  &  &  \\ \hline
            \cc 1 & 2 &  &  &  \\ \hline
            4 & 0 & 1 & 2 & \cc 3 \\ \hline
            2 & 3 & 4 & \cc 0 & 1 \\ \hline
        }\\
        \Downarrow
        \\
        \LSfive{
            0 & 1 & \cc 2 &  &  \\ \hline
            1 & \cc 2 &  &  &  \\ \hline
            \cc 2 & 3 &  &  &  \\ \hline
            3 & 4 & 0 & 1 & \cc 2 \\ \hline
            4 & 0 & 1 & \cc 2 & 3 \\ \hline
        }
        \hspace{1.5em}
        \LSfive{
            0 & 1 & \cc 2 &  &  \\ \hline
            3 & \cc 4 &  &  &  \\ \hline
            \cc 1 & 2 &  &  &  \\ \hline
            4 & 0 & 1 & 2 & \cc 3 \\ \hline
            2 & 3 & 4 & \cc 0 & 1 \\ \hline
        }
    \end{gather*}
\end{frame}

\section{Перебор конфигураций}
\begin{frame}{Полный перебор}

    \centering
    \begin{tabular}{|r|r|}\hline
        \rowcolor{green}\textcolor{white}{\;$n$\;} &\bf \textcolor{white}{Количество конфигураций} \\ \hline
        3 & 126 \\ \hline
        4 & 11\,440 \\ \hline
        5 & 2\,042\,975 \\ \hline
        7 & 262\,596\,783\,764 \\ \hline
        8 & 159\,518\,999\,862\,720 \\ \hline
        9 & 128\,447\,994\,798\,305\,325 \\ \hline
        11 & 169\,758\,547\,725\,351\,091\,518\,726 \\ \hline
        13 & 495\,874\,093\,230\,232\,452\,749\,553\,398\,586 \\ \hline
        16 & 8\,283\,675\,595\,268\,374\,292\,919\,471\,912\,522\,442\,632\,960 \\ \hline
    \end{tabular}

\end{frame} 


\section{Результаты}
\begin{frame}{Программа}

    \begin{block}{Программа}
        \begin{itemize}
            \item Написана система распределённых вычислений на добровольной основе, состоящая из серверной и клиентской части
            \item Для реализации алгоритма использованы Apache MXNet и NumPy
        \end{itemize}
    \end{block}
  
\end{frame} 


\begin{frame}{Доказанные свойства}

    \begin{block}{Доказанные свойства}
        \begin{itemize}
            \item Если конфигурация восстанавливается, то восстанавливается конфигурация являющаяся её частью
            \item Приведенная конфигурация размера $2n-3$ и меньше дополняется до конфигурации размера либо $2n-2$, либо $2n-1$
            \item $n+2$ неизвестных элементов восстанавливаются однозначно
            \item $2n-1$ неизвестных элементов, расположенных в двух строках или в двух столбцах, восстанавливаются однозначно
            \item Полная линейная система попарно ортогональных латинских квадратов порядка $n$ и эквивалентная ей при наличии не более $2n - 1$ неизвестных элементов однозначно восстанавливается в своём классе эквивалентности.
        \end{itemize}
    \end{block}
  
\end{frame} 


\begin{frame}{Полученные результаты}

    \begin{block}{Полученные результаты}
        \begin{itemize}
            \item Изучены системы попарно ортогональных латинских квадратов и доказан ряд утверждений об их восстановлении
            \item Разработан алгоритм восстановления таких систем
            \item Доказана математическая корректность алгоритма
            \item На основе исследованных свойств достигнуто ускорение исходного алгоритма и вычислен коэффициент ускорения
            \item Подтверждена гипотеза о восстановлении системы попарно ортогональных латинских квадратов порядка $n\in\{3,4,5,7,8\}$
        \end{itemize}
    \end{block}
  
\end{frame} 

\section{Список публикаций}
\begin{frame}{Публикации}

    \begin{block}{}
        \begin{itemize}
            \item Данилко В. Р. Восстановление системы попарно ортогональных латинских квадратов по частичной информации~// Мальцевские чтения. Международная конференция (Новосибирск, 19--22 ноября 2018 г.). Новосибирск, 2018 г. С. 87.
            \item Gorkunov E. V., Danilko V. R. Reconstructing sets of Latin squares, linear and equivalent to linear codes~//
            Proc. XVI Int. Symp. ``Problems of Redundancy in Information and Control Systems'' (Moscow, Russia, Oct.~21--25, 2019). 2019. P.~47--51.
        \end{itemize}
    \end{block}
  
\end{frame} 


\begin{frame}[standout]
    
    \Huge\centering
    \textcolor{white}{\bf Спасибо за внимание!}
  
\end{frame} 


\end{document}
